%%%%%%%%%%%%%%%%%%%%%%%%%%%%%%%%%%%%%%%%%
% Arsclassica Article
% LaTeX Template
% Version 1.1 (1/8/17)
%
% This template has been downloaded from:
% http://www.LaTeXTemplates.com
%
% Original author:
% Lorenzo Pantieri (http://www.lorenzopantieri.net) with extensive modifications by:
% Vel (vel@latextemplates.com)
%
% License:
% CC BY-NC-SA 3.0 (http://creativecommons.org/licenses/by-nc-sa/3.0/)
%
%%%%%%%%%%%%%%%%%%%%%%%%%%%%%%%%%%%%%%%%%

%----------------------------------------------------------------------------------------

%----------------------------------------------------------------------------------------

\documentclass[
10pt, % Main document font size
a4paper, % Paper type, use 'letterpaper' for US Letter paper
oneside, % One page layout (no page indentation)
%twoside, % Two page layout (page indentation for binding and different headers)
headinclude,footinclude, % Extra spacing for the header and footer
BCOR5mm, % Binding correction
]{scrartcl}

\usepackage{braket}
\usepackage{minted}
\usepackage{amsmath} % for '\text' macro
\usepackage{tabularx}

\usepackage[sorting=none]{biblatex}
\addbibresource{sample.bib}

%%%%%%%%%%%%%%%%%%%%%%%%%%%%%%%%%%%%%%%%%
% Arsclassica Article
% Structure Specification File
%
% This file has been downloaded from:
% http://www.LaTeXTemplates.com
%
% Original author:
% Lorenzo Pantieri (http://www.lorenzopantieri.net) with extensive modifications by:
% Vel (vel@latextemplates.com)
%
% License:
% CC BY-NC-SA 3.0 (http://creativecommons.org/licenses/by-nc-sa/3.0/)
%
%%%%%%%%%%%%%%%%%%%%%%%%%%%%%%%%%%%%%%%%%

%----------------------------------------------------------------------------------------
%	REQUIRED PACKAGES
%----------------------------------------------------------------------------------------

\usepackage[
nochapters, % Turn off chapters since this is an article        
beramono, % Use the Bera Mono font for monospaced text (\texttt)
eulermath,% Use the Euler font for mathematics
pdfspacing, % Makes use of pdftex’ letter spacing capabilities via the microtype package
dottedtoc % Dotted lines leading to the page numbers in the table of contents
]{classicthesis} % The layout is based on the Classic Thesis style

\usepackage{arsclassica} % Modifies the Classic Thesis package

\usepackage[T1]{fontenc} % Use 8-bit encoding that has 256 glyphs

\usepackage[utf8]{inputenc} % Required for including letters with accents

\usepackage{graphicx} % Required for including images
\graphicspath{{Figures/}} % Set the default folder for images

\usepackage{enumitem} % Required for manipulating the whitespace between and within lists

\usepackage{lipsum} % Used for inserting dummy 'Lorem ipsum' text into the template

\usepackage{subfig} % Required for creating figures with multiple parts (subfigures)

\usepackage{amsmath,amssymb,amsthm} % For including math equations, theorems, symbols, etc

\usepackage{varioref} % More descriptive referencing

%----------------------------------------------------------------------------------------
%	THEOREM STYLES
%---------------------------------------------------------------------------------------

\theoremstyle{definition} % Define theorem styles here based on the definition style (used for definitions and examples)
\newtheorem{definition}{Definition}

\theoremstyle{plain} % Define theorem styles here based on the plain style (used for theorems, lemmas, propositions)
\newtheorem{theorem}{Theorem}

\theoremstyle{remark} % Define theorem styles here based on the remark style (used for remarks and notes)

%----------------------------------------------------------------------------------------
%	HYPERLINKS
%---------------------------------------------------------------------------------------

\hypersetup{
%draft, % Uncomment to remove all links (useful for printing in black and white)
colorlinks=true, breaklinks=true, bookmarks=true,bookmarksnumbered,
urlcolor=webbrown, linkcolor=RoyalBlue, citecolor=webgreen, % Link colors
pdftitle={}, % PDF title
pdfauthor={\textcopyright}, % PDF Author
pdfsubject={}, % PDF Subject
pdfkeywords={}, % PDF Keywords
pdfcreator={pdfLaTeX}, % PDF Creator
pdfproducer={LaTeX with hyperref and ClassicThesis} % PDF producer
} % Include the structure.tex file which specified the document structure and layout

\hyphenation{Fortran hy-phen-ation} % Specify custom hyphenation points in words with dashes where you would like hyphenation to occur, or alternatively, don't put any dashes in a word to stop hyphenation altogether

%----------------------------------------------------------------------------------------
%	TITLE AND AUTHOR(S)
%----------------------------------------------------------------------------------------

\title{\normalfont\spacedallcaps{DMHP PROJECT REPORT}} % The article title

% \subtitle{Mani Nandadeep (IMT2019051), R Prasannavenkatesh (IMT2019063) \& Vijay Jaisankar (IMT2019525)} % Uncomment to display a subtitle

%\author{\spacedlowsmallcaps{Mani Nandadeep}} 
%\author{\spacedlowsmallcaps{R Prasannavenkatesh}}

\author{
  Mani Nandadeep\\
  \texttt{IMT2019051}
  \and
  R Prasannavenkatesh\\
  \texttt{IMT2019063}
  \and
  Vijay Jaisankar\\
  \texttt{IMT2019525}
  \and
  M Dhanush\\
  \texttt{IMT2019049}
}

% The article author(s) - author affiliations need to be specified in the AUTHOR AFFILIATIONS block

\date{} % An optional date to appear under the author(s)

%----------------------------------------------------------------------------------------

\begin{document}

%----------------------------------------------------------------------------------------
%	HEADERS
%----------------------------------------------------------------------------------------

\renewcommand{\sectionmark}[1]{\markright{\spacedlowsmallcaps{#1}}} % The header for all pages (oneside) or for even pages (twoside)
%\renewcommand{\subsectionmark}[1]{\markright{\thesubsection~#1}} % Uncomment when using the twoside option - this modifies the header on odd pages
\lehead{\mbox{\llap{\small\thepage\kern1em\color{halfgray} \vline}\color{halfgray}\hspace{0.5em}\rightmark\hfil}} % The header style

\pagestyle{scrheadings} % Enable the headers specified in this block

%----------------------------------------------------------------------------------------
%	TABLE OF CONTENTS & LISTS OF FIGURES AND TABLES
%----------------------------------------------------------------------------------------

\maketitle % Print the title/author/date block

\setcounter{tocdepth}{3} % Set the depth of the table of contents to show sections and subsections only

\tableofcontents % Print the table of contents

\newpage


\begin{section}{Stored Procedures}

A \textit{Stored Procedure} is a set of SQL statements with an assigned names, so that it can be shared with multiple applications and can generate dynamic queries as compared to a static query. \\ \\
It is \textbf{abstracted} from the regular code and just called in the API, similar to a function call and returns a cut of the database which are filtered according to the call arguments. \\ \\ 
Stored Procedure for the $tbl\_training$, $tbl\_districtexpense$, $tbl\_hrdatainfo$, $tbl\_districtmanasadhara$ and $tbl\_mnsalloaction$ tables have been implemented in this project. \\ \\ 
Now, rather than creating many API routes, a \textbf{single API route} will suffice for both training and district expense Data modules. Also this makes the graphing process simpler. \\ \\
Here is a list of all the \textit{stored procedures} implemented, their \textit{primary functionalities} and their \textit{argument lists}: 

\newpage

\begin{subsection}{getTraining}
This stored procedure generates a dynamic cut of the $tbl\_training$ table. \\ \\ \\
It filters the output based on the following parameters: 
\begin{itemize}
        \item Select a predefined list of columns.
        \item Group the output by a predefined list of columns.
        \item Filter by District, Facility, Event, Targeted Group, and Resource IDs.
        \item Reject all entries lying outside of a given date range.
        \item Group the output annually, monthly, or quarterly.
        \item Group the output by calendar year or financial year.
\end{itemize} 
\\ \\ \\

These are the arguments passed to it: 
\begin{itemize}
        \item display
        \item group\_by
        \item district\_list
        \item facility\_list
        \item event\_list
        \item target\_group\_list
        \item resource\_list
        \item start\_date
        \item end\_date
        \item timeperiod\_type
        \item year\_type
\end{itemize}

\end{subsection}

\newpage

\begin{subsection}{getMnsAlloAction}
This stored procedure generates a dynamic cut of the $tbl\_mnsalloaction$ table. \\ \\ \\
It filters the output based on the following parameters: 
\begin{itemize}
        \item Select a predefined list of columns.
        \item Group the output by a predefined list of columns.
        \item Filter by District, Financial Year, and Quarter IDs.
        \item Select arithmetic operations performed on a select list of columns.
\end{itemize} 
\\ \\ \\
These are the arguments passed to it: 
\begin{itemize}
        \item display
        \item group\_by
        \item agg
        \item district\_list
        \item quarterly\_list
        \item financial\_year
\end{itemize}

\end{subsection}

\newpage

\begin{subsection}{getHRData}
This stored procedure generates the rolling sum of active and contracted workers extracted from the $tbl\_hrdatainfo$ table. \\ \\ \\
It filters the output based on the following parameters: 
\begin{itemize}
        \item Group the output by a predefined list of columns.
        \item Filter by District and Taluka IDs.
        \item Reject all entries lying outside of a given date range.
\end{itemize} 
\\ \\ \\

These are the arguments passed to it: 
\begin{itemize}
        \item district\_list
        \item taluka\_list
        \item start\_date
        \item end\_date
\end{itemize}

\end{subsection}

\newpage

\begin{subsection}{getDistrictManasadhara}
This stored procedure generates a dynamic cut of the $tbl\_districtmanasadhara$ table. \\ \\ \\
It filters the output based on the following parameters: 
\begin{itemize}
        \item Select a predefined list of columns.
        \item Group the output by a predefined list of columns.
        \item Filter by District and Status IDs.
        \item Reject all entries lying outside of a given date range.
        \item Select arithmetic operations performed on a select list of columns.
        \item Group the output annually, monthly, or quarterly.
        \item Group the output by calendar year or financial year.
\end{itemize} 
\\ \\ \\
These are the arguments passed to it: 
\begin{itemize}
        \item display
        \item group\_by
        \item agg
        \item district\_list
        \item status\_list
        \item start\_date
        \item end\_date
        \item timeperiod\_type
        \item year\_type
\end{itemize}

\end{subsection}

\newpage


\begin{subsection}{getDistrictExpense}
This stored procedure generates a dynamic cut of the $tbl\_districtexpense$ table. \\ \\ \\
It filters the output based on the following parameters: 
\begin{itemize}
        \item Select a predefined list of columns.
        \item Group the output by a predefined list of columns.
        \item Filter by District and Status IDs.
        \item Reject all entries lying outside of a given date range.
        \item Select arithmetic operations performed on a select list of columns.
        \item Group the output annually, monthly, or quarterly.
        \item Group the output by calendar year or financial year.
\end{itemize} 
\\ \\ \\
These are the arguments passed to it: 
\begin{itemize}
        \item display
        \item group\_by
        \item agg
        \item district\_list
        \item status\_list
        \item start\_date
        \item end\_date
        \item timeperiod\_type
        \item year\_type
\end{itemize}

\end{subsection}




\end{section}

\newpage

\begin{section}{Parameters}

\begin{center}
\begin{tabular}{| c | c |}
\hline 
$display$ & This column specifies the list of columns that will be included in the query \\ \\ 
\hline 
 $group\_by$ & The resulting columns will be grouped by these values \\ \\
\hline 
$agg$ &  This column specifies the list of columns that will be included in the aggregation query \\ \\ 
\hline 
 $district\_list$ &  The output will contain values with DistrictIDs present in this list\\ \\ 
\hline 
$start\_date$ & This column specifies the starting point of date filtration \\ \\
\hline 
$end\_date$ & This column specifies the culmminating point of date filtration  \\ \\
\hline 
$timeperiod\_type$  & This specifies whether the query is ordered monthly, quarterly, or yearly  \\ \\ 
\hline 
$year\_type$  & Specifies what type of output is generated - financial year or calendar year \\ \\ 
\hline 
$status\_list$  & The output will contain values with StatusIDs present in this list \\ \\ 
\hline 
$taluka\_list$ & The output will contain values with TalukaIDs present in this list \\ \\ 
\hline 
$quaterly\_list$ & The output will contain values with DistrictIDs present in this list \\ \\ 
\hline 
 $financial\_year$ & This is a date proxy for the getMNSAlloAction stored procedure  \\ \\ 
\hline 
$facility\_list$ &  The output will contain values with FacilityTypeId present in this list\\ \\ 
\hline 
 $event\_list$ & The output will contain values with EventIDs present in this list   \\ \\ 
\hline 
$target\_group_list$ & The output will contain values with TargetGroupIDs present in this list \\ \\ 
\hline 
$resource\_list$ & The output will contain values with ResourceIDs present in this list  \\ \\ 
\hline 
\end{tabular} 
\end{center}

\end{section}


\newpage 

\begin{table}[ht]
\centering
\resizebox{\textwidth}{!}{\begin{tabular}{ll}
 \hline 
$display$ & This column specifies the list of columns that will be included in the query \\ \\ 
\hline 
 $group\_by$ & The resulting columns will be grouped by these values \\ \\
\hline 
$agg$ &  This column specifies the list of columns that will be included in the aggregation query \\ \\ 
\hline 
 $district\_list$ &  The output will contain values with DistrictIDs present in this list\\ \\ 
\hline 
$start\_date$ & This column specifies the starting point of date filtration \\ \\
\hline 
$end\_date$ & This column specifies the culmminating point of date filtration  \\ \\
\hline 
$timeperiod\_type$  & This specifies whether the query is ordered monthly, quarterly, or yearly  \\ \\ 
\hline 
$year\_type$  & Specifies what type of output is generated - financial year or calendar year \\ \\ 
\hline 
$status\_list$  & The output will contain values with StatusIDs present in this list \\ \\ 
\hline 
$taluka\_list$ & The output will contain values with TalukaIDs present in this list \\ \\ 
\hline 
$quaterly\_list$ & The output will contain values with DistrictIDs present in this list \\ \\ 
\hline 
 $financial\_year$ & This is a date proxy for the getMNSAlloAction stored procedure  \\ \\ 
\hline 
$facility\_list$ &  The output will contain values with FacilityTypeId present in this list\\ \\ 
\hline 
 $event\_list$ & The output will contain values with EventIDs present in this list   \\ \\ 
\hline 
$target\_group_list$ & The output will contain values with TargetGroupIDs present in this list \\ \\ 
\hline 
$resource\_list$ & The output will contain values with ResourceIDs present in this list  \\ \\ 
\hline 
\end{tabular}}
\end{table} 

\newpage




%----------------------------------------------------------------------------------------
%	BIBLIOGRAPHY
%----------------------------------------------------------------------------------------
%\section{References}
\renewcommand{\refname}{\spacedlowsmallcaps{References}} % For modifying the bibliography heading
% \medskip

%Sets the bibliography style to UNSRT and imports the 
%bibliography file "samples.bib".
% \bibliographystyle{unsrt}
%\bibliography{sample}

% ACM Notations

\printbibliography

%----------------------------------------------------------------------------------------

\end{document}
